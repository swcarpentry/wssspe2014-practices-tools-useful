\documentclass[submission,copyright,creativecommons]{eptcs}
\providecommand{\event}{WSSSPE 2, 2014} % Name of the event you are submitting to
\usepackage{breakurl}             % Not needed if you use pdflatex only.
\usepackage{color,soul}

\newcommand{\fixme}[1]{\hl{\#FIXME: #1}}

\title{Which Sustainable Software Practices\\ Do Scientists Find Most Useful?}

\author{
    Jory Schossau
    \institute{Michigan State University}
    \email{jory@msu.edu}
    \and
    Greg Wilson
    \institute{Software Carpentry}
    \email{gvwilson@software-carpentry.org}
}

\def\titlerunning{Which Sustainable Software Practices Do Scientists Find Most Useful?}
\def\authorrunning{J.D. Schossau, G.V. Wilson}
\begin{document}
\maketitle

\begin{abstract}
This is a sentence in the abstract.
This is another sentence in the abstract.
This is yet another sentence in the abstract.
This is the final sentence in the abstract.
\end{abstract}

\section{Introduction}

Education about sustainable software practices (SSP)
is fundamentally important to actual adoption.
Unfortunately,
discussion about sustainable practices with the average scientist often founders
because most scientists aren't using even \textit{basic} practices.
First encounters must therefore typically focus on concepts such as
when and where to use a script instead of a spreadsheet,
how to automate file manipulations typically performed by hand,
why to use a version control system to track work,
and how to write code for sanity and posterity.

Teaching these topics is the main aim of Software Carpentry,
a volunteer organization through which scientists (and others)
with the right knowledge and teaching skills can teach other scientists.
Its two-day workshops,
typically held at universities,
are just long enough to introduce learners to the handful of core competencies mentioned earlier.
Participants are typically excited about the subjects they learn,
but at present,
we can only make informed guesses about
which practices are actually useful for these scientists in their daily work,
which practices they are actually adopting (which is not the same thing),
and how effectively we are teaching them.
To answer these questions,
we need data.

\section{Surveys}

In the summer of 2013
the lead author designed a survey
to be administered to attendees directly before and after a Software Carpentry workshop.
This survey has let us gather information about
how well the workshop prepared attendees to handle sample real-world problems in the core competency areas,
so that we could see if what we set out to teach was being taught.
We decided early on to include only one skill-related question per core competency;
while this means we gather less (and less accurate) information per subject,
both research and experience have shown that shorter surveys are more likely to be completed.

The first set of questions on the survey was unique to the the pre-workshop survey.
These questions covered basic demographics such as career stage,
time spent programming,
and operating system preference,
and were used to help tailor workshops to the audience.
These questions revealed that most of our learners had little to no experience programming,
but that particular workshops would sometimes be filled with people who were more proficient.
It has proven very useful for instructors to know even this much about their learners.

\fixme{figures}

The second set of questions asked about familiarity with a topic,
and then,
if applicable,
about self-reported ability to solve a small real-world problem.
The difference between the pre-workshop and post-workshop survey responses allowed us to measure the effect of the workshop.
However, since asking people the same question twice doesn't yield a good measure of ability,
the post-workshop survey was created with similar but different questions to avoid likely recognition bias.

\fixme{figures}

Figure \fixme{fignum} shows changes in participants' pre-workshop familiarity with using the Unix shell.
We expected this distribution to be highly right-skewed because:

\begin{enumerate}
\item 
many attendees didn't have a computational background, and

\item 
workshops involved hands-on lessons with the shell.
\end{enumerate}

Figure \fixme{fignum} shows changes in skill level with the shell after the workshop.
We didn't have any preconceived expectations, \fixme{sure we did}
but found it encouraging that most people increased by at least one level.

\fixme{more to say if I remake the figure with a better query/filter}

\section{Interviews}

At the same time as we began gathering survey data,
the first author developed and began using a semi-structured protocol for interviews
in order to understand what practices scientists were finding useful six months after a workshop.
24 participants were interviewed concerning their work,
workshop experience,
how Software Carpentry helped them,
and how they might change the experience.
The transcripts of these interviews were analyzed
using the standard qualitative technique of open, axial, and selective coding,
which helped determine how best to translate qualitative responses into quantitative categorical data.

\fixme{figure 3 - coding from interviews}

We were initially surprised that Python was mentioned most often as the most useful topic rather than the command line.
The other core competencies followed in popularity of helpfulness,
which was more or less expected,
but some other fairly popular categories emerged during the coding process which were completely unanticipated.
These categories had been labeled "knowing", "psychology", and "resources".

\begin{itemize}
    \item
    Knowing represents the satisfaction attendees have expressed that they had been told about a technology or method
    for which they had no immediate use.
    This can include simply knowing about a core competency without any planned use,
    but also includes knowing about certain software or how types of scripting work.
    
    \item
    Psychology represents the usefulness attendees have expressed of learning about the psychology of programming,
    including empirically validated insights about how people work with computers,
    need breaks,
    or should write code.
    The second author (who taught several of the workshops from which survey participants were drawn)
    frequently cites this information while teaching a workshop.
    
    \item
    Resources represents attendees' mention of learning about books, websites, videos, and other sources of information
    not directly used within the workshop,
    but explicitly highlighted for their educational value.
    \fixme{Throw in some quotes so the reader knows what I'm talking about}
\end{itemize}

One participant's story of benefit from Software Carpentry stood out.
Their neighborhood was being repeatedly targeted by a small group of thieves,
so they decided to use their newfound skills from Software Carpentry to build an inexpensive surveillance camera.
They used an Arduino computer with storage and camera,
and created a Python script to manage the device,
all versioned with Git.
The footage obtained from this system allowed local law enforcement to catch and sentence the thieves.
We obviously can't expect everyone to benefit quite this much,
but this example does show how broadly useful even two days' worth of training can be.

The most unexpected result from our study was a nearly unanimous opinion about what should change about the workshops:
more depth,
more exercises,
and more about working with data files and plotting.
This finding about the need for scientists to know more about how to simply work with data
falls into a long line of informal and formal feedback,
and has led to the launch of a sibling project called Data Carpentry.

\section{Future Work}

As well as surveying workshop participants,
we are now collecting data from our instructors
to find out what technologies they used,
what topics they covered,
how they think it went,
and how well they think the least skilled attendee could perform on the hypothetical problems.
We will match this data with how the participants answered to see if there was anything unusual,
such as a disconnect in perceived instructional pacing,
or to see if there are trends of topics that are difficult for participants to grasp
and difficult for the instructors to measure their comprehension.

\fixme{the next paragraph says we're \textbf{going} to do 6-month studies, but didn't we say above that we're already doing them?}
\fixme{I thought we wanted to turn interviews @ 6mo, into atuomated surveys @ 6mo.}

Interviews give us a lot of insight,
but are labor-intensive and hence unscalable.
Our next goal is therefore to construct and validate better pre- and post-workshop questionnaires
so that we can track skill uptake and participant satisfaction
more systematically on a larger scale.

\section{Bibliography}


\nocite{*}
\bibliographystyle{eptcs}
\bibliography{references}
\end{document}
